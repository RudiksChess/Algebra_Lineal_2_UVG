%%%%%%%%%%%%%%%%%%%%%%%%%%%%%%%%%%%%%%%%%
% The Legrand Orange Book
% LaTeX Template
% Version 2.4 (26/09/2018)
%
% This template was downloaded from:
% http://www.LaTeXTemplates.com
%
% Original author:
% Mathias Legrand (legrand.mathias@gmail.com) with modifications by:
% Vel (vel@latextemplates.com)
%
% License:
% CC BY-NC-SA 3.0 (http://creativecommons.org/licenses/by-nc-sa/3.0/)
%
% Compiling this template:
% This template uses biber for its bibliography and makeindex for its index.
% When you first open the template, compile it from the command line with the 
% commands below to make sure your LaTeX distribution is configured correctly:
%
% 1) pdflatex main
% 2) makeindex main.idx -s StyleInd.ist
% 3) biber main
% 4) pdflatex main x 2
%
% After this, when you wish to update the bibliography/index use the appropriate
% command above and make sure to compile with pdflatex several times 
% afterwards to propagate your changes to the document.
%
% This template also uses a number of packages which may need to be
% updated to the newest versions for the template to compile. It is strongly
% recommended you update your LaTeX distribution if you have any
% compilation errors.
%
% Important note:
% Chapter heading images should have a 2:1 width:height ratio,
% e.g. 920px width and 460px height.
%
%%%%%%%%%%%%%%%%%%%%%%%%%%%%%%%%%%%%%%%%%

%----------------------------------------------------------------------------------------
%	PACKAGES AND OTHER DOCUMENT CONFIGURATIONS
%----------------------------------------------------------------------------------------

\documentclass[11pt,fleqn]{book} % Default font size and left-justified equations

\input{structure.tex} % Insert the commands.tex file which contains the majority of the structure behind the template

%\hypersetup{pdftitle={Title},pdfauthor={Author}} % Uncomment and fill out to include PDF metadata for the author and title of the book

%----------------------------------------------------------------------------------------

\begin{document}

%----------------------------------------------------------------------------------------
%	TITLE PAGE
%----------------------------------------------------------------------------------------

\begingroup
\thispagestyle{empty} % Suppress headers and footers on the title page
\begin{tikzpicture}[remember picture,overlay]
\node[inner sep=0pt] (background) at (current page.center) {\includegraphics[width=\paperwidth]{Pictures/background.pdf}};
\draw (current page.center) node [fill=blue!30!white,fill opacity=0.6,text opacity=1,inner sep=1cm]{\Huge\centering\bfseries\sffamily\parbox[c][][t]{\paperwidth}{\centering Álgebra Lineal 2\\[15pt] % Book title
{\Large Una aventura en las matemáticas}\\[20pt] % Subtitle
{\huge Rudik Roberto Rompich}}}; % Author name
\end{tikzpicture}
\vfill
\endgroup

%----------------------------------------------------------------------------------------
%	COPYRIGHT PAGE
%----------------------------------------------------------------------------------------

\newpage
~\vfill
\thispagestyle{empty}

\noindent Copyright \copyright\ 2020 Rudik Rompich\\ % Copyright notice

\noindent \textsc{Published by Rudiks}\\ % Publisher

\noindent \textsc{rudiks.com}\\ % URL

\noindent Licensed under the Creative Commons Attribution-NonCommercial 3.0 Unported License (the ``License''). You may not use this file except in compliance with the License. You may obtain a copy of the License at \url{http://creativecommons.org/licenses/by-nc/3.0}. Unless required by applicable law or agreed to in writing, software distributed under the License is distributed on an \textsc{``as is'' basis, without warranties or conditions of any kind}, either express or implied. See the License for the specific language governing permissions and limitations under the License.\\ % License information, replace this with your own license (if any)

\noindent \textit{First printing, October 2020} % Printing/edition date

%----------------------------------------------------------------------------------------
%	TABLE OF CONTENTS
%----------------------------------------------------------------------------------------

%\usechapterimagefalse % If you don't want to include a chapter image, use this to toggle images off - it can be enabled later with \usechapterimagetrue

\chapterimage{Pictures/0002} % Table of contents heading image

\pagestyle{empty} % Disable headers and footers for the following pages

\tableofcontents % Print the table of contents itself

\cleardoublepage % Forces the first chapter to start on an odd page so it's on the right side of the book

\pagestyle{fancy} % Enable headers and footers again

%----------------------------------------------------------------------------------------
%	PART
%----------------------------------------------------------------------------------------
\part{Tercer parcial}

\chapterimage{Pictures/0001} % Chapter heading image

\chapter{Parcial 3}

\section{Funcionales lineales}\index{Funcionales lineales}

\begin{definition}
Sea $V$ un espacio vectorial sobre $\mathbb{F}$. Entonces, una transformación lineal $$f:V\mapsto\mathbb{F}$$ es un funcional lineal sobre $V$. 
\end{definition}

\begin{exercise}
Sean $\alpha_1,...,\alpha_n \in\mathbb{R}$. Definimos:
$$\Phi:\mathbb{R}^n\mapsto\mathbb{R}\ni$$
$$\Phi(x_1,...,x_n)=\alpha_{1}x_{1}+...+\alpha_{n}x_{n}$$
$\Rightarrow\Phi$ es funcional lineal. 
\end{exercise}
\begin{notation}
$$f:\mathbb{R}^2\mapsto\mathbb{R}\ni$$
$$\Phi(x,y)=2x-y$$
\end{notation}

\begin{exercise}
Sea $C[0,1]$ un conjunto de funciones continuas en [2,1] y considere:
$$T:C[0,1]\mapsto\mathbb{R}\ni$$
$$T(g)=\int^1_0 g(x)dx$$
Nótese si $f,g\in[0,1]$ y $\alpha\in R$$\Rightarrow T(\alpha f+g)=\int^1_0[\alpha f+g](x)dx=\int^1_0[(\alpha f)(x)+g(x)]dx=\alpha\int^1_0f(x)dx+\int^1_0f(x)dx+\int^1_0 g(x)dx=\alpha T[f]+T[g]\Rightarrow$ es lineal $\Rightarrow$ T es funcional lineal. 
\end{exercise}

\begin{exercise}
Sea $$d:\mathbb{R}^{nxn}\mapsto\mathbb{R}\ni$$
$$d(A)=\text{determinante de A}$$
Recordar que: $$det(A+B)\neq det(A)+det(B)$$
$$det(\alpha A)\neq \alpha det(A)$$
$$d(A) \text{ no es funcional lineal}$$
\end{exercise}

\begin{exercise}
Sea $$T:\mathbb{R}^{nxn}\mapsto\mathbb{R}\ni$$
$$T(A)=\text{ traza de A}$$
$$\text{Si } A=[a_{ij}]\Rightarrow Tr(A) = \sum^n_{i=1}a_{ij}$$
$$\Rightarrow Tr(A) \text{ es funcional lineal.}$$
\end{exercise}
\begin{exercise}
Sea $V$ el espacio de todas las funciones sobre $\mathbb{R}$.
Definimos $$C_t: V\mapsto\mathbb{R}\ni$$
$$C_t(f)=f(t)\text{ , donde t es un número fijo.}$$
Nótese que:
\begin{enumerate}
    \item Sea $f,g\in V\Rightarrow L_t[f+g]=(f+g)(t)=L_t(f)+L_t(g)$
    \item Sea $\alpha\in\mathbb{R}\mapsto L_t(\alpha f)=(\alpha f)(t)=\alpha f(t)=\alpha L_t(f)\Rightarrow\text{ Es funcional lineal.}$
\end{enumerate}
\end{exercise}

\begin{notation}
Considere el funcional lineal $$f:\mathbb{R}^n\mapsto\mathbb{R}\ni$$$$f(x_1,...,x_n)=\alpha_1 x_1+...+\alpha_n x_n.$$
$$ \alpha_1\in\mathbb{R}$$ (fijos)
Sea $B=\{e_1,...,e_n\}$ la base usual de $\mathbb{R}^n$ y sea $B'=\{1\}$ la base usual de $\mathbb{R}$. $$f(1,0,...,0)=\alpha_1(1)+\alpha_2(0)+...+\alpha_1(0)$$
$$=\alpha_1$$
$$f(0,1,0,...,0)=\alpha_2$$
$$\vdots$$
$$\Rightarrow[f]^{B'}_B =[\alpha_1\,\alpha_2\,...\,\alpha_n]$$

\end{notation}

\begin{notation}
Si $f$ son funcionales lineales. 
$$f\in f[V,\mathbb{F}]\text{ si dim(V)=n}$$
$$\Rightarrow dim( f[V,\mathbb{F}])= n\cdot1=n$$
\end{notation}

\begin{definition}[$V^{*}$]
Al espacio de funciones lineales de V es $\mathbb{V}$ es $\mathbb{F}$ se le llama al espacio dual de $V$.
\end{definition}

\begin{notation}
Si $dim(V)=n \Rightarrow dim(V^*)=n$
\end{notation}

\subsection{Teorema}
\begin{theorem}
Sea $V$ un espacio vectorial finito dimensional y $B=\langle v_1,...,v_n\rangle$ una base ordenada de $V$. Entonces, existe una base: $B^* = \{\Phi_1,...,\Phi_n\}$ de $V^*$, tal que: 
$$\Phi_i(V_j)=
    \begin{cases} 
      1 & i=j \\
      0 & i\neq j  
   \end{cases} = \delta_{ij} $$
$$\Phi_i(V_j)=\delta_{ij}\xleftarrow{\text{ Delta de Kronecker}}$$


\end{theorem}

\subsection{Ejercicio}
\begin{exercise}
Considere la base de $\mathbb{R}^2$,
$$B=\{(2,1),(3,1)\}$$
entonces, encuentre una base para para $(\mathbb{R^2})^{*}\xleftarrow{\mathcal{L}[\mathbb{R}^2,\mathbb{R}^2]}$
\end{exercise}

\textbf{Solución:}
\begin{align}
    B^* &= \{\phi_1,\phi_2\} & \text{es tal que:}\\
    \intertext{Debemos encontrar $\alpha_1,\alpha_2,\beta_1,\beta_2$}
    \phi_1(x,y)&=\alpha_1 x+\alpha_2 y\\ 
    \phi_2(x,y)&=\beta_1 x+\beta_2 y\\
    \intertext{Encontramos $\alpha_1 y \alpha_2$:}
    \phi_1(v_1)&=\phi(2,1)=2\alpha_1+\alpha_1=1  & \delta_{11}\\
    \phi_1(v_2)&=\phi(3,1)=3\alpha_1+\alpha_2=0  & \delta_{11}\\
    \implies &\alpha_1=-1, \alpha=3\\
    \implies & \phi_1(x,y)=-x+3y\\
    \intertext{Encontramos $B_1,B_2$:}
    \phi_2(v_1)=\phi_2(2,1)= 2\beta_1+\beta_2 =0\\ 
    \phi_2(v_2)=\phi_2(3,1)= 3\beta_1+\beta_2 =1\\
    \implies & \beta_1=1,\beta_2=-2\\
    \implies &\phi_2(x,y) =x -2y 
\end{align}
$\implies$ La base dual de $B$, denotada por $B^*$ (i.e. la base del espacio dual $V^)*$, es \\$B^{*}=\{-x+3y,-x+2y\}$

\subsection{Ejercicio}
\begin{exercise}
Dada la base de $\mathbb{R}^3$:\\
$$B=\{(1,-1,3),(0,1,-1),(0,3,-2)\},$$
encuentre la base dual de $B^*$ (i.e. la basa para $\mathcal{L}[\mathbb{R}^3,\mathbb{R}]$)
\end{exercise}

\newcommand{\al}{\alpha}
\newcommand{\be}{\beta}
\newcommand{\g}{\gamma}
\newcommand{\p}{\phi}
\newcommand{\up}{\upsilon}
\begin{align}
    \phi_1(x,y,z)&=\alpha_1 x +\alpha_2 y+\alpha_3 z\\
    \phi_2(x,y,z)&=\beta_1 x +\beta_2 y+\beta_3 z\\
    \phi_3(x,y,z)&=\gamma_1 x +\gamma_2 y+\gamma_3 z\\
    \intertext{Encontramos $\alpha_1,\alpha_2,\alpha_3$: }
    \p_1(v_1)&=\p_1(1,-1.3)=\al_1-\al_2+2\al_3=1\\
    \p_2(v_2)&=\p_1(0,1,-1)=0\al_1+\al_2 -1\al_3=0\\
    \p_3(v_3)&= \p_1(0,3,-2)=0\al_1+3\al_2-2\al_3=0\\
    \implies &\al_1=1,\al_2=\al_3=0 \implies \p_1(x,y,z)=x
    \intertext{Encontramos $\be_1,\be_2,\be_3$:}
    \p_2(v_1)&=\p_2(1,-1,3)=1\be-1\be+2\be= 0\\
    \p_2(v_2)&=\p_2(0,1,-1)=0\be+1\be-1\be= 1\\
    \p_2(v_3)&=\p_2(0,3,-2)=0\be+3\be-2\be= 0\\
    \implies &\be_1=7,\be_2=-2,\be_3=-3 \implies \p_2(x,y,z)=7x-2y-3z
    \intertext{Encontramos $\g_1,\g_2,\g_3$}
    \p_3(v_1)&=\p_3(1,-1,3)=1\g-1\g+2\g= 0\\
    \p_3(v_2)&=\p_3(0,1,-1)=0\g+1\g-1\g= 1\\
    \p_3(v_3)&=\p_3(0,3,-2)=0\g+3\g-2\g= 0\\
    \implies &\g_1=2,\g_2\g_3=1 \implies \p_3(x,y,z)=-2x+y+z
    \intertext{Por lo tanto:}
    B^* &= \{x,7x-2y-3z,-2x+y+z\}
    \intertext{Por otra parte, es necesario probar que $B^*$ es linealmente independiente}
    \text{Considere: }& \up_1\p_1+\up_2\p_2+...+\up_n\p_n =0\\
    \text{A probar: }& \up_1=\up_2=...=\up_n=0\\
    \intertext{Sea $v_i\in B$, $1\leq i\leq n$}
    \implies & (\up_1\p_1+...+\up_1\p_i+...+\up_n\p_n)(v_i)=0(v_i)\\
    \implies & \up_1\p_1(v_i)+...+\up_1\p_i(v_i)+...+\up_n\p_n)(v_i)=0\\
    \implies & 0+1 +0 &= 0\\
    \implies &\up_1=\up_2=...=\up_n=0 \implies \{\p_1,...,\p_n\}\text{ es linealmente independiente}\\
\end{align}

\subsection{Teorema}
\begin{theorem}
Sea $V$ un espacio vectorial finito dimensiona y sea $B=\{x_1,...,x_2\}$ una base ordenada para $V$. Entonces existe una base $B^* = \{\p_1,...,\p_n \}$ para $V^* \ni$ $$\p_i(x_j)=\delta_ij$$
Además, 
\begin{align}
    (i) \forall \p \in V^*, \text{ se tiene:}\\
    \p =\p(x_1)\p_1+\p(x_2)\p_2+...+\p(x_n)\p_n\\
    (ii) \forall x \in V, \text{se tiene que:}\\
    x=\p_1(x)x_1+\p_2(x)x_2+...+\p_n(x)x_n\\
\end{align}
\end{theorem}
\begin{proof}
\begin{align}
    \intertext{A probar: $B^*$ es linealmente independiente. Considere:}
    \al_1\p_1+...+\al_n\p_n ?
\end{align}
\end{proof}
%----------------------------------------------------------------------------------------
%	PART
%----------------------------------------------------------------------------------------


%----------------------------------------------------------------------------------------
%	INDEX
%----------------------------------------------------------------------------------------

\cleardoublepage % Make sure the index starts on an odd (right side) page
\phantomsection
\setlength{\columnsep}{0.75cm} % Space between the 2 columns of the index
\addcontentsline{toc}{chapter}{\textcolor{blue}{Index}} % Add an Index heading to the table of contents
\printindex % Output the index

%----------------------------------------------------------------------------------------

\end{document}
